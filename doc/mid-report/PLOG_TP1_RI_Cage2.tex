\documentclass[a4paper,11pt]{article}
\usepackage[utf8]{inputenc}
\usepackage{lmodern}
\usepackage[T1]{fontenc}
\usepackage[babel=true]{microtype}
\usepackage[portuguese]{babel}
\usepackage[pdftex]{hyperref}
\usepackage{graphicx}
\usepackage{eurosym}
\usepackage{scrextend}
\usepackage{hyphenat}
\usepackage{url}

\begin{document}

\begin{titlepage}
\title{\huge \textbf{Cage\\[1cm] \Large Relatório intercalar\\[1cm]
\includegraphics[width=50mm,scale=0.5]{logo.png}\\[1cm] \large Programação em
Lógica\\[0.25cm] \small $3^o$ ano\\[0.05cm]Mestrado Integrado em Engenharia Informática e
Computação\\[1.7cm]}\normalsize Turma 4 - Grupo Cage\_2}
%### 
\author{Jose Peixoto \\Luís Cruz \and 200603103\\201303248 \and ei12134@fe.up.pt
\\ up201303248@fe.up.pt\\[1cm]}
\maketitle
\pagestyle{empty} % titlepage must not be numbered
\end{titlepage}

\section{Descrição do jogo}
\subsection{História}
\subsection{Regras}

\section{Representação do estado do jogo}
\iffalse (tipicamente uma lista de listas que incluem diferentes átomos para as
peças), com exemplificação em Prolog de estados iniciais, intermédios e finais do jogo, acompanhados de imagens ilustrativas \fi

\section{Visualização do tabuleiro em modo de texto}
\iffalse (tipicamente uma lista de listas que incluem diferentes átomos para as
peças), com exemplificação em Prolog de estados iniciais, intermédios e finais do jogo, acompanhados de imagens ilustrativas \fi

\section{Movimentos}
\iffalse (tipos de jogadas) possíveis, definindo os cabeçalhos dos predicados
que serão depois implementados. \fi

\begin{thebibliography}{9}
\bibitem{lamport93}
  Sterling, Leon
  \emph{The Art of Prolog},
  The MIT Press
  2nd edition,
  2000.
  \bibitem{bl}Cage rules,
  \url{http://www.marksteeregames.com/Cage_rules.html}, 14 10 2016.
\end{thebibliography}

\end{document}