\documentclass[a4paper,11pt]{article}
\usepackage[utf8]{inputenc}
\usepackage{lmodern}
\usepackage[T1]{fontenc}
\usepackage[babel=true]{microtype}
\usepackage[portuguese]{babel}
\usepackage[pdftex]{hyperref}
\usepackage{graphicx}
\usepackage{eurosym}
\usepackage{scrextend}
\usepackage{hyphenat}
\usepackage{url}
\usepackage{hyperref}
\usepackage{float}

\begin{document}

\hypersetup{pageanchor=false}
\begin{titlepage}
\title{\huge \textbf{Cage\\[1cm] \Large Relatório intercalar\\[1cm]
\includegraphics{res/logo.png}\\[1cm] \large Programação em
Lógica\\[0.25cm] \small $3^o$ ano\\[0.05cm]Mestrado Integrado em Engenharia Informática e
Computação\\[1.7cm]}\normalsize Turma 4 - Grupo Cage\_2}

\author{José Peixoto \\Luís Cruz \and 200603103\\201303248 \and ei12134@fe.up.pt
\\ up201303248@fe.up.pt\\[1cm]}
\maketitle
\pagestyle{empty} % titlepage must not be numbered
\end{titlepage}
\hypersetup{pageanchor=true}

\section{Descrição do jogo}
O Cage é um jogo de estratégia em tabuleiro parecido com as damas que foi inventado por Mark Steere em maio de 2010. O autor descreve-o como um jogo para dois jogadores sem qualquer informação oculta. É abstracto e sem factor sorte, sem empates e sem temas. É jogado num tabuleiro de damas 10x10 ou 8x8 e ao contrário do jogo original das damas, no início todo tabuleiro está preenchido com peças promovidas ``damas''. ``Jogo de aniquilação de alta energia'' é a frase escolhida pelo autor para caricaturizar o jogo, uma vez que o movimento para o centro do tabuleiro assegura a aniquilação de pelo menos uma das cores.

\subsection{Regras}
O Cage é jogado por dois jogadores num tabuleiro de damas com 50 damas vermelhas e 50 damas azuis na versão de tabuleiro 10x10 ou com 32 damas vermelhas e 32 damas azuis na versão de 8x8 tabuleiro. O tabuleiro é inicializado preenchendo todas as casas com damas de cor alternada.

\begin{figure}[H]
    \center
    \includegraphics[scale=0.7]{res/1-initial-setup.jpg}
    \caption{Estado inicial do tabuleiro}
    \label{fig:1-initial-setup.png}
\end{figure}

\subsubsection{Objetivo}
Para vencer é necessário capturar todas as damas inimigas. No final, pode ganhar-se mesmo que se perca a última peça que se está a movimentar (saltar) para capturar todas as damas últimas inimigas.

\subsubsection{Movimentos}
Existem quatro tipos de movimentos:
\begin{enumerate}
  \item Restrito
  \item Centralizador
  \item Adjunto
  \item Salto
\end{enumerate}
Durante um turno, um jogador apenas pode utilizar um tipo de movimento.

\paragraph{Restrição 1}
Nunca se pode colocar uma dama ortogonalmente (horizontal ou verticalmente) adjacente a uma dama de cor idêntica. Nem de forma transitória durante um turno de vários movimentos.

\begin{figure}[H]
    \centering
    \includegraphics[scale=0.5]{res/2-restriction-1.jpg}
    \caption{1ª restrição}
    \label{fig:2-restriction-1.jpg}
\end{figure}

\paragraph{Restrição 2}
Nunca se pode movimentar uma dama que tenha adjacências ortogonais com damas inimigas para uma casa onde tal não aconteça.

\begin{figure}[H]
    \centering
    \includegraphics[scale=0.4]{res/2-restriction-2.jpg}
    \caption{2ª restrição}
    \label{fig:2-restriction-2.jpg}
\end{figure}

\paragraph{Centralizador}
Este movimento de uma casa, permite à dama deslocar-se na horizontal, vertical ou diagonal para uma casa vazia e que permite que a dama se aproxime do centro do tabuleiro.

\begin{figure}[H]
    \centering
    \includegraphics[scale=0.4]{res/3-centering.jpg}
    \caption{Movimento centralizador}
    \label{fig:3-centering.jpg}
\end{figure}

\paragraph{Adjacente}
Uma dama que nao tenha adjacencias ortogonais com damas inimigas pode-se mover uma casa em qualquer direção que  contenha adjacencias ortogonais com uma ou mais damas inimigas.

\paragraph{Salto}
O movimento de salto permite capturar uma dama inimiga, movimentando a dama do jogador de uma casa ortogonalmente adjacente de um lado da dama inimiga para a casa vazia adjacente do lado oposto. É possivel capturar uma dama enimiga nas casas periféricas do tabuleiro de uma casa adjacente e do lado oposto da dama inimiga na borda do tabuleiro. O resultado é que quer a dama capturada quer a dama que captura são removidas do tabuleiro neste tipo de captura para fora dos limites do tabuleiro.

\begin{figure}[H]
    \centering
    \includegraphics[scale=0.55]{res/5-jump.jpg}
    \caption{Salto}
    \label{fig:5-jump.jpg}
\end{figure}

\section{Representação do estado do jogo}
\iffalse (tipicamente uma lista de listas que incluem diferentes átomos para
as peças), com exemplificação em Prolog de estados iniciais, intermédios e finais do jogo, acompanhados de imagens ilustrativas \fi
\subsection{Representação do estado inicial do tabuleiro:}
\begin{verbatim}
\end{verbatim}

\section{Visualização do tabuleiro em modo de texto}
\iffalse (tipicamente uma lista de listas que incluem diferentes átomos para as peças), com exemplificação em Prolog de estados iniciais, intermédios e finais do jogo, acompanhados de imagens ilustrativas \fi

\section{Movimentos}
\iffalse (tipos de jogadas) possíveis, definindo os cabeçalhos dos predicados que serão depois implementados. \fi
De uma forma geral, nos predicados dos movimentos a seguir apresentados é requerida a receção das linhas e colunas iniciais e finais da peça a mover e do tabuleiro.
\subsection{Restrições}
Restrições aos predicados dos movimentos declarados a seguir. Presume-se que a implementação dos outros predicados relativos aos restantes movimentos requerem a avaliação satisfatória das seguintes restrições.
\begin{verbatim}
restriction_1(row,col,adj_row,adj_col,boad).
restriction_2(row,col,adj_row,adj_col,board).
\end{verbatim}

\subsection{Adjacente}
Cabeçalho do predicado do movimento para uma casa adjacente.
\begin{verbatim}
adjoining_mode(row,col,dest_row,dest_col,board).
\end{verbatim}

\subsection{Salto}
Cabeçalho do predicado do movimento em salto que permite capturar damas inimigas.
\begin{verbatim}
jump(row,col,dest_row,dest_col,board).
\end{verbatim}

\subsection{Centralizador}
Cabeçalho do predicado do movimento que centraliza a dama ao centro do tabuleiro.
\begin{verbatim}
centering_move(row,col,dest_row,dest_col,board).
\end{verbatim}

\begin{thebibliography}{9}
\bibitem{lamport93}
  Sterling, Leon
  \emph{The Art of Prolog},
  The MIT Press
  2nd edition,
  2000.
  
  \bibitem{bl}Abstract games,
  \url{http://www.marksteeregames.com/MSG_abstract_games.html}, 14 10 2016.
  
  \bibitem{bl}Cage rules,
  \url{http://www.marksteeregames.com/Cage_rules.html}, 14 10 2016.
\end{thebibliography}

\end{document}