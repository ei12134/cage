\documentclass[a4paper]{article}

%use the english line for english reports
%usepackage[english]{babel}
\usepackage[portuguese]{babel}
\usepackage[utf8]{inputenc}
\usepackage{indentfirst}
\usepackage{graphicx}
\usepackage{verbatim}


\begin{document}

\setlength{\textwidth}{16cm}
\setlength{\textheight}{22cm}

\title{\Huge\textbf{Jogo Cage}\linebreak\linebreak\linebreak
\Large\textbf{Relatório Final}\linebreak\linebreak
\linebreak\linebreak
\includegraphics[scale=0.1]{feup-logo.png}\linebreak\linebreak
\linebreak\linebreak
\Large{Mestrado Integrado em Engenharia Informática e Computação} \linebreak\linebreak
\Large{Programação em Lógica}\linebreak
}

\author{\textbf{Grupo 2:}\\ José Peixoto - 200603103 \\ Luís Cruz - 201303248 \\\linebreak\linebreak \\
 \\ Faculdade de Engenharia da Universidade do Porto \\ Rua Roberto Frias, s\/n, 4200-465 Porto, Portugal \linebreak\linebreak\linebreak
\linebreak\linebreak\vspace{1cm}}
%\date{Junho de 2007}
\maketitle
\thispagestyle{empty}

%************************************************************************************************
%************************************************************************************************

\newpage

\section*{Resumo}
Resumo sucinto do trabalho com 150 a 250 palavras (problema abordado, objetivo, como foi o problema resolvido/abordado, principais resultados e conclusões).

\newpage

\tableofcontents

%************************************************************************************************
%************************************************************************************************

%*************************************************************************************************
%************************************************************************************************

\newpage

%%%%%%%%%%%%%%%%%%%%%%%%%%
\section{Introdução}

Descrever os objetivos e motivação do trabalho. Descrever num parágrafo breve a estrutura do relatório.


%%%%%%%%%%%%%%%%%%%%%%%%%%
\section{O Jogo Cage}

O Cage é um jogo de estratégia em tabuleiro semelhante às damas que foi inventado por Mark Steere em maio de 2010. O autor descreve-o como um jogo para dois jogadores sem qualquer informação oculta. É um jogo abstrato sem fator de sorte nem empates. É jogado num tabuleiro de damas 10x10 ou 8x8 e, ao contrário do jogo original das damas, todo tabuleiro está preenchido, no início, com peças já promovidas a ``damas''. ``Jogo de aniquilação de alta energia'' é a frase escolhida pelo autor para caricaturar o jogo, uma vez que o movimento para o centro do tabuleiro assegura a aniquilação, de pelo menos, uma das cores.


\subsection{Regras}
O Cage é jogado por dois jogadores num tabuleiro de damas com 50 damas vermelhas e 50 damas azuis na versão de tabuleiro 10x10 ou com 32 damas vermelhas e 32 damas azuis na versão de 8x8 tabuleiro. O tabuleiro é iniciado preenchendo todas as casas com damas de cor alternada.

\subsubsection{Objetivo}
Para vencer é necessário capturar todas as damas inimigas. No final, pode ganhar-se mesmo que se perca a última peça que se está a movimentar (saltar) para capturar todas as damas inimigas ainda em jogo.

\subsubsection{Movimentos}
Existem quatro tipos de movimentos:
\begin{enumerate}
  \item Restrito
  \item Centralizador
  \item Adjacente
  \item Salto
\end{enumerate}
Durante um turno, um jogador apenas pode utilizar um tipo de movimento.

\paragraph{Restrição 1}
Nunca se pode colocar uma dama ortogonalmente (horizontal ou verticalmente) adjacente a uma dama de cor idêntica. Nem de forma transitória durante um turno de vários movimentos.

\paragraph{Restrição 2}
Nunca se pode movimentar uma dama que tenha adjacências ortogonais com damas inimigas para uma casa onde tal não aconteça.

\paragraph{Centralizador}
Este movimento de uma casa, permite à dama deslocar-se na horizontal, vertical ou diagonal para uma casa vazia e que permite que a dama se aproxime do centro do tabuleiro.


\paragraph{Adjacente}
Uma dama que não tenha adjacências ortogonais com damas inimigas pode mover-se apenas uma casa em qualquer direção que contenha adjacências ortogonais com uma ou mais damas inimigas.

\paragraph{Salto}
O movimento de salto permite capturar uma dama inimiga, movimentando a dama do jogador de uma casa ortogonalmente adjacente de um lado da dama inimiga para a casa vazia adjacente do lado oposto. É possível capturar uma dama inimiga nas casas periféricas do tabuleiro de uma casa adjacente e do lado oposto da dama inimiga na borda do tabuleiro. O resultado é que quer a dama capturada quer a dama que captura são removidas do tabuleiro.


%%%%%%%%%%%%%%%%%%%%%%%%%%
\section{Lógica do Jogo}

Descrever o projeto e implementação da lógica do jogo em Prolog, incluindo a forma de representação do estado do tabuleiro e sua visualização, execução de movimentos, verificação do cumprimento das regras do jogo, determinação do final do jogo e cálculo das jogadas a realizar pelo computador utilizando diversos níveis de jogo. Sugere-se a estruturação desta secção da seguinte forma:

\subsection{Representação do Estado do Jogo} Pode ser idêntico ao descrito no relatório intercalar.)

\subsection{Visualização do Tabuleiro} (Pode ser idêntico ao descrito no relatório intercalar.)

\subsection{Lista de Jogadas Válidas} Obtenção de uma lista de jogadas possíveis. Exemplo: \textit{valid\_moves(+Board, -ListOfMoves)}.

\subsection{Execução de Jogadas} Validação e execução de uma jogada num tabuleiro, obtendo o novo estado do jogo. Exemplo: \textit{move(+Move, +Board, -NewBoard)}.

\subsection{Avaliação do Tabuleiro} Avaliação do estado do jogo, que permitirá comparar a aplicação das diversas jogadas disponíveis. Exemplo: \textit{value(+Board, +Player, -Value)}.

\subsection{Final do Jogo} Verificação do fim do jogo, com identificação do vencedor. Exemplo: \textit{game\_over(+Board, -Winner)}.

\subsection{Jogada do Computador} Escolha da jogada a efetuar pelo computador, dependendo do nível de dificuldade. Por exemplo: \textit{choose\_move(+Level, +Board, -Move)}.


%%%%%%%%%%%%%%%%%%%%%%%%%%
\section{Interface com o Utilizador}

Descrever o módulo de interface com o utilizador em modo de texto.


%%%%%%%%%%%%%%%%%%%%%%%%%%
\section{Conclusões}
Que conclui deste projecto? Como poderia melhorar o trabalho desenvolvido?


\clearpage
\addcontentsline{toc}{section}{Bibliografia}
\renewcommand\refname{Bibliografia}
\bibliographystyle{plain}
\bibliography{myrefs}

\newpage
\appendix
\section{Nome do Anexo}
Código Prolog implementado devidamente comentado e outros elementos úteis que não sejam essenciais ao relatório.

\end{document}
